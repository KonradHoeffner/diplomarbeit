\chapter{Verwandte Arbeiten}\label{relatedwork}
%http://www.dcs.shef.ac.uk/~sam/stringmetrics.html -> string metriken

%Preliminary Results in Tag Disambiguation using DBpedia
In \citet{tag_disambiguation_dbpedia} werden wie in dieser Arbeit DBpedia-Entitäten als Kandidaten mit einem Ähnlichkeitsmaß des Information Retrieval disambiguiert.
Dabei dienen jedoch keine ganzen Sätze sondern Tags aus einer Tagmenge als Eingabedaten.
Die Berechnung der Ähnlichkeit erfolgt nicht über semantische Informationen in Form der diesen Kandidaten zugeordneten Tripel genutzt, sondern über eine Zuordnung von
weiteren Tags, die als Kontext auf einen bestimmten Sinn eines Tags auftreten.

\citet{cucerzan07} beschreibt eine Disambiguierung mit Named Entities (Eigennamen) basierend auf Wikipedia-Daten.
Die dort verwendeten Testdaten sind frei verfügbar\footnote{\url{http://research.microsoft.com/en-us/um/people/silviu/WebAssistant/TestData/}} und werden daher zur Verifikation der Disambiguierungsmethode in dieser Arbeit verwendet.
Es handelt sich dabei um die je zwei zuoberst angezeigten Artikel auf der Webseite von MSNBC\footnote{\url{http://www.msnbc.msn.com/}} am 2. Januar 2007 in den damaligen zehn Nachrichtenkategorien
Wirtschaft, U.S.-Politik, Unterhaltung, Gesundheit, Sport, Wissenschaft und Technik, Reisen, Fernsehnachrichten, Nachrichten über die USA und weltweite Nachrichten.
In diesen Artikeln wurden 797 Named Entities (Eigennamen) bestimmt und als Oberflächenformen zur Bestimmung von Bedeutungskandidaten genutzt, welche 
in Form von Wikipedia-Artikelnamen vorliegen. Diese Kandidaten dienen nun als Ausgangspunkt für eine Disambiguierung, die auf 
den den Namen der Wikipediaartikeln, den Redirects und Disambiguierungsseiten basiert.
Um die Disambiguierung zu evaluieren, existiert eine manuell erstellte Zuordnung der Oberflächenformen zu ihren korrekten Kandidaten, wobei 755 dieser Oberflächenformen dazu geeignet sind
und in 650 Fällen diese Kandidaten auch in der Wikipedia auf dem verwendeten Stand vom September 2006 existieren, was einem Anteil von \valunit{81.5558}{\%} entspricht.
Das dort beschriebene System erreichte diesen Testdaten eine Genauigkeit von \valunit{91.4}{\%}.
In DBpedia 3.5, die in dieser Arbeit zur Evaluierung verwendet wird, existieren hingegen 651 dieser Kandidaten, wobei jedoch 143 davon Redirects sind, vermutlich weil sich der Name dieser Artikel 
inzwischen geändert hat. In diesen Fällen wurde das Ziel des Redirects als gültige Bedeutung definiert.

% and the proposed system could not hypothesize a
% disambiguation for them. The accuracy on the re-
% maining 5,131 surface forms was 86.2% for the
%                                                           91.4%, versus a
% baseline system and 88.3% for the proposed sys-
%                                                     51.7%
% tem. A McNemar test showed that the difference is
% not significant, the main cause being that the ma-
% jority of the test surface forms were unambiguous.
% When restricting the test set only to the 1,668 am-
% biguous surface forms, the difference in accuracy
% between the two systems is significant at p = 0.01.
% An error analysis showed that the Wikipedia set
% used as gold standard contained relatively many
% surface forms with erroneous or out-of-date links,
% many of them being correctly disambiguated by
% the proposed system (thus, counted as errors). For
% example, the test page “The Gods (band)” links to
% Paul Newton, the painter, and Uriah Heep, which is
% a disambiguation page, probably because the origi-
% nal pages changed over time, while the proposed
% system correctly hypothesizes links to Paul New-
% ton (musician) and Uriah Heep (band).

%Dabei werden jedoch keine semantischen 
