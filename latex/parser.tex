
\iffalse
Unglücklicherweise konnte keine Ontologie für das Penn Treebank Tag Set gefunden werden. Aufgrund des bereits großen Umfanges dieser Arbeit und des Umstandes, das der Autor kein ausgebildeter Sprachwissenschaftler ist,
 erschien es auch nicht praktikabel, solch eine Ontologie selbst zu erstellen.
Das Umlernen eines POS-Taggers auf ein anderes Tagset ist beim Stanford Parser jedoch möglich, wenn man ihn mit POS-annotierten Trainingstexten versorgt.\footnote{\url{http://nlp.stanford.edu/software/parser-faq.shtml#f}}
Es war also nur noch eine beliebige für das Englische geeignete Ontologie gesucht, für die es POS-annotierten Text gibt.
Die Wahl fiel dabei auf das Susanne\footnote{Surface and underlying structural analysis of natural English}-Tagset.
Sowohl das Susanne Corpus, eine passende Treebank\footnote{\url{www.grsampson.net/Resources.html}} als auch die Ontologie\footnote{\url{http://141.89.100.105/owl-docu/susa.html}} sind frei verfügbar.

In der FAQ des Stanford Parsers wird ein Satz angegeben, der das erwartete Format deutlich macht:

\begin{verbatim}
`The/DT quick/JJ brown/JJ fox/NN jumped/VBD over/IN the/DT lazy/JJ dog/NN ./.
\end{verbatim}

Die Treebank besteht jedoch aus mehreren Dateien, welche Zeilen wie diese enhalten:
\begin{verbatim}
G12:0030.27	-	PPHS1m	He	he	[S[Nas:s.Nas:s]
G12:0030.30	-	VBDZ	was	be	[Vsu.
G12:0030.33	-	VVGv	waiting	wait	.Vsu]S]
G12:0030.39	-	YF	+.	-	.O]
\end{verbatim}
Das gewünschte Format kann daraus jedoch erzeugt werden.

Zusammenführen aller relevanten Dateien in Eine:
\begin{verbatim}
find | sed "s/\.\///" |
egrep -v "lexicon|documentation|\." |
xargs cat >>allinone.txt
\end{verbatim}
Auswählen der relevanten Spalten:
\begin{verbatim}
cat allinone.txt | cut -f3,4 > allinone_tag_and_word.csv
\end{verbatim}
Vertauschen der Spalten und Einsetzen des Trennzeichens:
\begin{verbatim}
cat allinone_tag_and_word.csv |
sed "s/\([^\t]*\)\t\([^\t]*\)/\2\/\1/"
> allinone_stanfordformat.txt
\end{verbatim}
Alle Zeilenumbrüche entfernen:
\begin{verbatim}
cat allinone_stanfordformat.txt |
tr "\n" " " > allinone_stanfordformat_oneline.txt
\end{verbatim}
Zeilenumbrüche nach dem Satzende hinzufügen um das endgültige Format zu bekommen:
\begin{verbatim}
cat allinone_stanfordformat_oneline.txt |
sed "s/+\.\/YF /+\.\/YF\\n/g"
> susanne_treebank_readyfortraining.txt
\end{verbatim}
\todo{was ist mit einem trainierten parser model für die susa.owl?}
\todo{was ist mit einem trainierten parser model für die susa.owl?}
\fi