\section{Programmstruktur Wortschatz2DBpedia}
\todo{entscheiden, wo das genau hinsoll mit dem paradigma und der kapselung des veränderlichen und so}
Ein Grundprinzip der Objektorientierten Programmierung ist "`kapsele was veränderlich ist und isoliere es von dem, was bleibt"'.
\subsection{Paketliste}

\begin{center}
\begin{tabular}{ll}
\toprule
Paketname\footnotemark	&Funktion\\
\hline
Standardpaket	&Verschiedene Helperprogramme\\
analyse		&Analyse einer annotierten Stichprobe\\
datasource	&Datenquellen für DBpedia und den Wortschatz\\
gui		&Graphisches Benutzerinterface für das Mapping\\
match		&Erstellen und Konfiguration des Mappings\\
\bottomrule
\end{tabular}
\end{center}
\footnotetext{Exklusive dem Präfix "`wortschatz2dbpedia."'}

Im Folgenden werden nun die wichtigsten Klassen und Methoden noch einmal separat aufgeführt.
Es handelt sich dabei um diejenigen, die für die normale Nutzung, die Wartung oder die Analyse relevant sind.
\subsection{Paket analyse}
\paragraph{Klasse Analyse}
Diese Klasse liest eine ausgewertete Stichprobe ein und wertet diese aus.
Mit Hilfe des gegebenen Klassifizierers wird die Stichprobe in Teilmengen unterteilt und für jede dieser Teilmengen die Fehlerrate bestimmt.
Die Entscheidung, ob ein bestimmter StringTransformator beim Mapping benutzt wird, lässt sich damit objektiv an der Fehlerrate der durch ihn erzeugten Links festmachen.
Die Ergebnisse werden auf folgende Art und Weise ausgegeben:
\begin{itemize}
\item Anzeige als Torten- (Vorkommenshäufigkeit jeder Klasse) und Balkendiagramm (Fehlerrate)
\item Speichern der Diagramme im Format encapsulated postscript (eps)
\item Speichern als Tabelle im Format comma separated value (csv) mit einem Tabulator als Trennzeichen
\end{itemize}
\paragraph{Interface Classfier}
Dieses Interface ist eine Anwendung des Strategy-Entwurfsmusters\footnotemark.
\footnotetext{Funktionsparameter sind in Java nicht möglich. Das Strategy-Entwurfsmuster kapselt daher einen Algorithmus (hier eine Implementierung von \emph{classify()}) als Klasse.}
\begin{lstlisting}
public interface Classifier
{
	public String[] getClasses();
	// class 0 will be excluded from further analysis  
	public int classify(SampleEntry entry);
}
\end{lstlisting}
\subparagraph{Anwendung}
\todo{umschreiben: Sollte eine 
Bei einer Änderung 
Zum Analysieren der Korrektheit einer bestimmten Untermenge einer Stichprobe.}
%\begin{bem}

%\end{bem}
\begin{bsp}
Wir möchten unser Mapping vergrößern.
Bis jetzt erzeugen wir nur dann einen Link, wenn DBpedia-Label und Wort exakt übereinstimmen.
Nun überlegen wir, dass man Abkürzungen wie "`ADIDAS"' ja auch "`A.D.I.D.A.S"' schreiben könnte. Also bauen wir einen neuen StringTransformator, der alle Punkte entfernt.
\begin{lstlisting}
package wortschatz2dbpedia.match;
public class RemovePeriodTransformer implements StringTransformer
{
  public String transform(String s)
  {return s.replaceAll("\.","");}
  
  public String getDescription()
  {return "Removes all periods.";}
} 
\end{lstlisting}
Nun möchten wir uns davon überzeugen, wie groß die erwartete Ausbeute und Fehlerquote dieses StringTransformators ist.
Daher benötigen wir eine annotierte Stichprobe. Ein Umfang von 200 Einträgen sollte dabei einen guten Kompromiss zwischen Auswertungszeit und Aussagekraft der Ergebnisse gewährleisten\footnotemark.
\footnotetext{\todo{hier was schreiben über konfidenzintervalle und so}}
Da die Auswertung einer solchen Stichprobe durchaus mehrere Tage in Anspruch nehmen kann, ist es unpraktisch, bei jeder Änderung eine neue Stichprobe zu erstellen.
Daher wurde zuerst ein möglichst unscharfes Mapping erstellt, dass auf möglichst hohe Coverage ausgelegt war und in dem möglichst alle zu testenden Klassen enthalten sein sollten.
Für dieses Mapping wurde dann eine Stichprobe mit 200 Einträgen erstellt und annotiert \footnote{im Projektordner unter \emph{analyse/mapping1/stichprobe\_ausgewertet.csv}}.
Wir möchten die Einträge in folgende Kategorien einteilen
\begin{itemize}
\item der Eintrag wird vom alten Transformator akzeptiert
\item der Eintrag wird nur vom neuen Transformator akzeptiert
\item der Eintrag wird von beiden nicht akzeptiert
\end{itemize}
Wir erstellen also den folgenden Klassifizierer:
\begin{lstlisting}
package wortschatz2dbpedia.analyse;
public class PeriodClassifier implements Classifier
{
  private final String[] classes =
  {"different","equal","equal except for periods"}; 	
  
  @Override
  public int classify(SampleEntry e)
  {
    if(e.resource.equals(e.label)) return 1;
    if  (e.resource.replaceAll("\\.", "").
     equals(e.label.replaceAll("\\.", ""))) return 2;
    return 0;
  }

  @Override public String[] getClasses() {return classes;}
}
\end{lstlisting}
In der Klasse Analyse setzen wir nun den neuen Klassifizierer ein:
\begin{lstlisting}
public static void main(String[] args) throws IOException
{
  SampleEntry[] sample = SampleEntry.readFromCSV("meineAusgewerteteStichprobe.csv");
  Classifier[] classifiers = {new PeriodClassifier()};
  ...
}
\end{lstlisting}
Das Programm zeigt 
\end{bsp}

\subparagraph*{Format der ausgewerteten Stichprobe}

\subsection{Paket datasource}
\subsection{Paket gui}
\subsection{Paket match}
