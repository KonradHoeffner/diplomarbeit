\section{Modul \textit{Domänenidentifizierung}}
\begin{center}
\begin{tabular}{l|l}
\multicolumn{2}{c}{Modulabhängigkeiten}\\
Vorraussetzung& Empfohlen\\
\hline
\textit{Wortschatz2DBpedia} & \textit{Disambiguation}
\end{tabular}

\end{center}

\subsection{Problemstellung}
Eine wichtiges Merkmal höherer Ordnung\todo{klingt das ok?} eines Textes, das zu seiner Klassifizierung beitragen kann, ist die Domäne, also der Gegenstand oder auch Fachgebiet des Textes.
Um zu einer große Menge von Texten eine passende Domäne zur Verfügung zu haben und um aus den im Text vorhandenen Konzepten auf einen gemeinsamen Gegenstand zu schließen, wird eine
umfangreiche taxonomische (Superklassen-) Hierarchie benötigt.
\todo{den ganzen satz vielleicht zum wortschatz2dbpedia-modul schieben} Durch das NLP2RDF-Modul \textit{Wortschatz2DBpedia} werden zu den im Text vorkommenden, explizit angegebenen Substantive\footnotemark 
\footnotetext{Da zum Zeitpunkt der Diplomarbeit keine Anapher-Resolution verfügbar ist, können indirekt angegebene Substantive ("`\textit{Der Präsident von Norwegen} bekam den Friedensnobelpreis."',
"`\textit{Er} bestellte noch einen Kaffee."' nicht aufgelöst - und damit auch nicht klassifiziert - werden.}
Referenzen zu den Wikipedia-Artikeln hinzugefügt, 
\subsection{YAGO}
\textit{YAGO}\cite{yago} ist eine große Ontologie von Weltwissen mit hoher Präzision und Coverage, die automatisch aus Wikipedia und WordNet erstellt wurde.
YAGO zeichnet sich dadurch aus, dass es die exzellente, von Hand gepflegte, Taxonomie von Wordnet mit der Vielfalt von Artikeln und damit Konzepten aus Wikipedia verbindet.

Zu jedem Wikipedia-Artikel enthält YAGO ein passendes Individuum. Dabei ist jeder Wikipedia-Artikeltitel ein Kandidat, ein YAGO-Individuum zu werden.

\begin{bsp}
"`The Enterprise is in space."'
Durch Ausführung von Wortschatz2DBpedia und einer Disambiguierung haben wir beispielhaft die referenzierten Wikipedia-Artikel \url{Starship_Enterprise} und \url{Outer_Space} extrahiert.
\begin{center}
\begin{tabular}{llll}
Wort		&Artikelname			&YAGO-Individuum\\
\hline
Enterprise	&\url{Starship_Enterprise}	&\url{Starship} \url{Enterprise}\\
space		&\url{Outer_Space}		&\url{outer} \url{space}\\
\end{tabular}

\begin{tabular}{llll}
YAGO-Individuum			&type			&subClassOf\\
\hline
\url{Starship} \url{Enterprise}	&Star Trek ships	&~\\
~				&craft			&~\\
~				&\ldots			&~\\
\url{outer} \url{space}		&~			&space\\
~				&~			&location\\
\end{tabular}
\end{center}


\end{bsp}

\todo{Describes-Relation einbauen}


Der Wikipedia-Seitentitel "`Starship Enterprise"' ist also ein Kandidat, das Individuum \url{Starship} \url{Enterprise} in YAGO zu werden.


Wir erhalten also aus einer Menge von Artikeln eine Menge von Individuen.
Für jedes dieser Individuen interessieren uns nun die zugehörige Klassen in der Hierarchie von YAGO.
Die Subklassenhierarchie wird in YAGO durch die Relation \textit{TYPE} ausgedrückt.

\url{Elephant} TYPE \url{Elephants}

Die YAGO-Klassenhierarchie hat die Gestalt eines \textit{DAG}s (directed acyclic graph, gerichteter azyklischer Graph)

%Mit Hilfe der \textit{YAGO}-Klassenhierarchie wird eine 

Durch die Ausführung von Wortschatz2DBpedia ist jedes Wort mit einer Menge an möglichen passenden Wikipedia-Artikeln ausgezeichnet. Wurde zusätzlich noch eine Disambiguation ausgeführt, dann ist dies für 
jedes Wort genau ein passender Wikipedia-Artikel.


\subsection{Extrahieren der YAGO-Klasse und der Klassenhierarchie}
Das Modell der YAGO-Ontologie\footnote{Download unter \url{http://www.mpi-inf.mpg.de/yago-naga/yago/yago.zip} (1 Gb)} benutzt dieselbe Wissensrepräsentation wie RDFS.
die 

Die Klassenhierarchie findet sich unter \url{facts/subClassOf/WordNetLinks.txt}.

Die Zuordnung von Wikipedia-Klassen zu Wordnet-Klassen findet sich unter \url{facts/subClassOf/ConceptLinker.txt}.
Durch Sortieren nach der Wordnet-Klasse lässt sich durch binäre Suche die Performance wesentlich erhöhen.

\url{facts/type/IsAExtractor.txt}